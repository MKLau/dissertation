\documentclass[12pt]{article}
\usepackage{color}
\usepackage{cite}
\usepackage{geometry}                % See geometry.pdf to learn the layout options. There are lots.
%\usepackage{pdflscape}        %single page landscape
                                %mode \begin{landscape} \end{landscape}
\geometry{letterpaper}                   % ... or a4paper or a5paper or ... 
%\usepackage[parfill]{parskip}    % Activate to begin paragraphs with an empty line rather than an indent
\usepackage{graphicx}
\usepackage{amssymb}
\usepackage{/Library/Frameworks/R.framework/Resources/share/texmf/Sweave}
\usepackage{chronology}
\newcommand{\etal}{\textit{et al.} }
\linespread{1.5}

\title{PROSPECTUS: \\ The Evolution of Ecological Networks}
\author{M.K. Lau}
%\date{}                                           % Activate to display a given date or no date

\begin{document}

\maketitle

\begin{chronology}[5]{1950}{1980}{1.90ex}{\textwidth}
\event{1950}{\tiny 1736: Euler solves K\"{o}nigsberg Bridge Problem}
\event{1951}{\tiny 1859: Darwin's "tangled bank" in \text{Origin of Species}}
\event[1953]{1953}{...}
\event{1957}{\tiny H.T. Odum: electrical circuit models}
\event{1969}{\tiny von Bertalanffy: "General Systems Theory"}
\event{1972}{\tiny May: network structure and stability}
\event{1980}{\tiny Pimm: "Bound's on food web connectance"}
\end{chronology}
\paragraph{}
\begin{chronology}[5]{1980}{2013}{1.90ex}{\textwidth}
\event{1987}{\tiny Jordano: "Patterns of mutualistic interactions..."}
\event{1999}{\tiny Sole: networks in macroevolution and extinction}
\event{2001}{\tiny Dunne: Food-web connectance and size}
\event{2003}{\tiny Bascompte: nestedness of mutualistic networks}
\event{2010}{\tiny Th\'{e}bault and Fontaine: network stability and architecture}
\end{chronology}

\pagebreak
\setcounter{secnumdepth}{-1}  %activate to start numbering from one
\setcounter{tocdepth}{3}  %%activate to number sections
\tableofcontents

\section{Introduction}
\begin{itemize}
\item QUESTION: How does ecological structure at the scale of
  communities arise from genetics?
\item APPROACH:
  \begin{itemize}
  \item Review the literature on:
    \begin{enumerate}
    \item evolution of interactions
    \item genetic contribution to interactions
    \item function and evolution of interaction structure in complex
      communities
    \end{enumerate}
  \item Generate a evolutionary network model to examine the
    implications of foundation species evolution on communities
    \begin{enumerate}
    \item Two models:
      \begin{enumerate}
      \item Intraspecific variation vs. Species Averages
      \item Associate species interactions (hierarchical and intransitive/diffuse) vs no interactions
      \end{enumerate}
    \item Defining the system boundaries for communities
    \item IBMs?, Mass-action?, random forests?
    \item Find the paper on modling interactions
    \end{enumerate}
  \item Examine the network structure of interactions between plants
    and their associated species and the interactions among associated
    species.
    \begin{enumerate}
    \item Arthropods (herbivores, predators, parasites)
    \item Understory plants
    \item Bark lichen
    \item Endophytic fungi
    \end{enumerate}
  \end{itemize}

\end{itemize}

\section{Ecological and evolutionary interaction network exploration}

\subsection{\textbf{Published as:} Lau\ MK, Whitham\ TG, Lamit\ LJ, Johnson\ NC
(2010) Ecological and Evolutionary Interaction Network Exploration:
Addressing the Complexity of Biological Interactions in Natural
Systems with Community Genetics and Statistics. JIFS 7:17-25}

\begin{itemize}
\item Ecological communities play an integral role in determining
  ecosystem functions. However, community-level patterns and processes
  are complex because they are typically comprised of many interacting
  components. Therefore, pairwise reductionist investigations  of
  interactions among species are unlikely to reveal the dynamics of
  the whole community. Here, we present results from a study of the
  interactions among members of a lichen community associated with
  different genotypes of a foundation tree species, \textit{Populus
    angustifolia}.  

\item Three key findings emerge. First, null-model based analysis of
  species co-occurrence patterns suggest that interactions are likely
  contributing to lichen community structure. Second, the pattern of
  co-occurrences and pairwise correlations of lichen species suggest
  that interactions among lichens are primarily facilitative. Third,
  the significance and magnitude of co-occurrence patterns vary among
  genotypes of \textit{P. angustifolia} suggesting that the strength of
  facilitative interactions among lichens is tree genotype dependent.  

\item In combination, direct and indirect plant genetic effects on the
  interactions of lichens appear to play an important role in defining
  the lichen community. We believe that a community genetics approach
  focused on foundation species will allow researchers to better
  understand the selection pressures that shape communities and that
  many unexpected outcomes will emerge. From this perspective we
  discuss future research directions that employ greater analytical
  power to further quantify the complex network of species
  interactions within communities.
\end{itemize}

\section{How do ecological networks evolve?}
\subsection{Summary}
  \begin{itemize}
  \item In his \textit{The Origin of Species} Darwin pondered a
    community as ``a tangled bank'' suggesting the complex structure
    of interactions among species and how it might contribute to
    evolutionary dynamics.
  \item It was not until the middle of the last century that
    a systems approach was developed to approach this complexity as a
    whole (von Bertalanffy 1968).
  \item Robert May's foundational investigations into the
    community-wide effects of interaction network structure set the
    stage for a systems approach in ecological community dynamics (May
    1972).
  \item Subsequent to May's work, researchers have investigated
    the network structure in empirical interaction networks, such as
    food-webs and plant-aninal networks (Cohen \etal 1977, Pimm 1979,
    Sugihara \etal 1989, Jordano 1987, Dunne \etal 2002).
  \item Both theoretical and empirical work has shown that genotypic
    variation in a foundation species influences the structure of
    communities; however, it is still unclear how whether this is due
    to overwhelming effects of the foundation species or interactions
    among species re-inforcing those effects.
  \item In this study we will extend the research of Shuster \etal
    2006 to examine how the structure of interactions among
    species will affect the outcome of genetic variation in a
    foundation species.
  \end{itemize}
\subsection{Methods}
\begin{itemize}
\item Using Shuster \etal 2006 community genetics simulation method,
  which uses a mass action model to determine the community
  composition based on allelic complimentarity, we will simulate
  herbivore abundances based on a range of underlying genetic
  structure of a foundation species.
\item This model will form the base on which we will apply three
  different models of interactions:
  \begin{enumerate}
  \item Intransitive -- interactions will be distributed among species
    with a high degree of feedbacks
  \item Trophic -- directed interactions will be distributed to form a
    directed hierarchy
  \item Modular -- sub-groupings of species will be created through a
    non-uniform distribution of interactions
  \end{enumerate}
\item We will then run a series of simulations in which allelic
  complimentarity will determine the efficacy of interactions
  generating the community structure of species associated with
  individuals of the simulated foundation species
\end{itemize}

\subsection{Outcomes}
\begin{itemize}
\item By exploring a range of networks of variable structure, we will
  be able to identifying possible mechanisms creating community
  structure at the scale of individual trees
\item In addition, we will better understand how genetic information
  flows through ecological communities
\end{itemize}


\section{Does plant genetics influence bark lichen interaction network
  structure?}
\subsection{Summary}
\begin{itemize}
\item Sample quadrats of bark lichen on replicate clones of
  cottonwood genotypes in the common garden
\item Generate network models using co-occurrence based methods
\item Test for the effects of genotype on overall network structure
  and network structural statistics
\end{itemize}

\subsection{Methods}
\begin{itemize}
\item 
\end{itemize}

\subsection{Outcomes}
\begin{itemize}
\item Lichen interaction network structure varies among genotypes
\item This variation is primarily due to:
  \begin{enumerate}
  \item number of interactions
  \item number of species
  \item centrality
  \item modularity
  \end{enumerate}
\end{itemize}

\section{How does intraspecific variation influence the
  structure of plant-arthropod (herbivore, predator, parasite)
  interaction networks?}
\subsection{Summary}
- build off of the findings of Fontaine et al. 2009, Thebault et al. 2010 and Fontaine et
al. 2011
- use plant-herbivore network database to conduct meta-analysis type
review and construct hypotheses
- altering scale of genetic variation from species to genotype

\subsection{Methods}
- INFORMATICS AND MODELING: Use common garden data from Gina, Art,
Sharon to understand the structure of plant-herbivore networks
- EXPERIMENT: Collect leaf modifer-inquiline data using natural
abundances and also a paper-clip manipulation experiment

\begin{itemize}
\item All data bipartite graph approach
\item Combined covariance modeling and trophic data
\item Use cross-type and genotype identity as the module identifier
\item Detect modularity in the network empirically
\item Separate by trophic information
\item Use phylogenetics to help resolve the trophic structure
  among arthropods
\item Gina, Art, Sharon, Randy, Bill, (Adrian?)
\end{itemize}

\subsection{Outcomes}
\begin{itemize}
\item You might have to divide up into two studies, one looking at
  the effect of hybridization and the other looking at the effect of
  genotypic variation.
\end{itemize}

%%%%%%%%%%%%%%%%%%%%%%%%%%%%%%%%%%%%%%%%%%%%%%%%%%%%%%%%%%%%%%%%%%%%%%%
%%%%%%%%%TALK WITH LAMIT ABOUT DOING THIS PROJECT WITH HIS DATA%%%%%%%%
%%%%%%%%%%%%%%%%%%%%%%%%%%%%%%%%%%%%%%%%%%%%%%%%%%%%%%%%%%%%%%%%%%%%%%%
%% \section{How does intraspecific variation influence the structure of
%%   plant-plant interaction networks?}
%% \begin{itemize}
%% \item QUESTION: 
%% \item APPROACH:
%%   \begin{itemize}
%%   \item Bipartite graph approach
%%   \item Use cross-type and genotype as a module identifier
%%   \item Lamit's garden data
%%   \end{itemize}
%% \item OUTCOMES:
%%   \begin{itemize}
%%   \item You might want to combine this study with the arthropod study
%%     if the results are clear enough and can be combined into one
%%     cohesive paper.
%%   \end{itemize}
%% \end{itemize}

%%%%%%%%%%%%%%%%%%%%%%%%%%%%%%%%%%%%%%%%%%%%%%%%%%%%%%%%%%%%%%%%%%%%%%%
%%%%%FIGURE OUT WHAT IS HAPPENING WITH MICHALET AND THE CO-TUTELLE%%%%%
%%%%%%%%%%%%%%%%%%%%%%%%%%%%%%%%%%%%%%%%%%%%%%%%%%%%%%%%%%%%%%%%%%%%%%%
%% \section{Network analysis of secies covariance structure}
%% \begin{itemize}
%% \item Build on co-occurrence analyses and network inference
%%   literature
%% \item Utilize phylogenetic data (Work with Brad)
%% \item How do alpine foundation species influence species
%%   covariance structure?
%% \end{itemize}

%% %%Figure construction
%% <<echo=false,results=hide,label=fig1,include=false>>=
%% @ 


%% %%Figure plotting
%% \begin{figure} 
%% \begin{center} 
%% <<label=fig1,fig=TRUE,echo=false>>=
%% <<fig1>> 
%% @ 
%% \end{center} 
%% \caption{}
%% \label{fig:one}
%% \end{figure}


%% %%Activate for bibtex vibliography
%% \cite{goossens93}
%% \bibliographystyle{plain}
%% \bibliography{/Users/Aeolus/Documents/bibtex/biblib}


\end{document}  


