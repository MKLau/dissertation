\documentclass[12pt]{article}
\usepackage{color}
\usepackage{cite}
\usepackage{geometry}                % See geometry.pdf to learn the layout options. There are lots.
%\usepackage{pdflscape}        %single page landscape
                                %mode \begin{landscape} \end{landscape}
\geometry{letterpaper}                   % ... or a4paper or a5paper or ... 
%\usepackage[parfill]{parskip}    % Activate to begin paragraphs with an empty line rather than an indent
\usepackage{graphicx}
\usepackage{amssymb}
\usepackage{Sweave}
\newcommand{\etal}{\textit{et al.}}
\usepackage{hyperref}  %\hyperref[label_name]{''link text''}
                       %\hyperlink{label}{anchor caption}
                       %\hypertarget{label}{link caption}
\linespread{1.5}

\title{Cottonwood Group Literature Review (1978-2012)}
\author{M.K. Lau}
%\date{}                                           % Activate to display a given date or no date

\begin{document}
\maketitle

\setcounter{tocdepth}{3}  %%activate to number sections
\tableofcontents

\section{2012}
\begin{itemize}
\item Ferrier, SM, RK Bangert, E Hersch-Green, JK Bailey, GJ Allan and TG
Whitham. 2012. Unique arthropod communities on different host-plant
genotypes results in greater arthropod diversity. Arthropod-Plant
Interactions In Press:0-0. abstract.

\item Meneses N, JK Bailey, GJ Allan, RK Bangert, MA Bowker, B Rehill, GM
Wimp, RL Lindroth, and TG Whitham. 2012. Arthropod community
similarity in clonal stands of aspen: A test of the genetic similarity
rule. Écoscience 19:48-58. abstract. fulltext PDF.
\end{itemize}

\section{2011}
\begin{itemize}
\item Adams, R.I.; S. Goldberry; T.G. Whitham; M.S. Zinkgraf and
R. Dirzo. 2011. Hybridization among dominant tree species correlates
positively with understory plant diversity. American Journal of Botany
98:1623-1632. abstract. This article is also available online.

\item Compson, Z.G.; K.C. Larson; M.S. Zinkgraf and T.G. Whitham. 2011. A
genetic basis for the manipulation of sink-source relationships by the
galling aphid, Pemphigus betae.. Oecologia
167:711-721. abstract. fulltext PDF.

\item David Solance Smith, Jen Schweitzer, Philip Turk, Joe Bailey, Stephen
Shuster and Thomas Whitham. 2011. Soil-mediated local adaptation
alters seedling survival and performance. Plant and Soil. Plant and
Soil in pressabstract.

\item David Solance Smith, Joseph K. Bailey, Stephen M. Shuster and Thomas
G. Whitham. 2011. A geographic mosaic of trophic interactions and
selection: trees, aphids and birds.. Journal of Evolutionary Biology
24:422-429. abstract. fulltext PDF.

\item L.J. Lamit, T. Wojtowicz, Z. Kovacs, S.C. Wooley, M. Zinkgraf,
T.G. Whitham, R.L. Lindroth, and C.A. Gehring. 2011. Hybridization
among foundation tree species influences the structure of associated
understory plant communities. Botany 89:165-174. abstract. fulltext
PDF.

\item LAMIT, L.J., M.A. BOWKER, L.M. HOLESKI, R. Reese NÆSBORG, S.C. WOOLEY,
M. ZINKGRAF, R.L. LINDROTH, T.G. WHITHAM,
C.A. GEHRING. 2011. Genetically-based trait variation within a
foundation tree species influences a dominant bark lichen. Fungal
Ecology 4:103-106. abstract. fulltext PDF.
 
\item Wymore, A.S., Keeley, A.T.H., Yturralde, K.M., Schroer, M.L., Propper,
C.R. and Whitham, T.G.. 2011. Genes to ecosystems: exploring the
frontiers of ecology with one of the smallest biological units. New
Phytologist 191:19-36. abstract. fulltext PDF.
\end{itemize}

\section{2010}
\begin{itemize}
\item Arnold, A.E., Lamit, L.J., Gehring, C.A., Bidartondo, M.I., Callahan,
H.. 2010. Interwoven branches of the plant and fungal trees of
life. New Phytologist 185:874-878. fulltext PDF.

\item Keith, A.R.; Bailey, J.K. and T.G. Whitham. 2010. A genetic basis to
community repeatability and stability. Ecology
91:3398-3406. abstract. fulltext PDF.

\item Lau, M.K., Whitham, T.G., Lamit, L.J., Johnson, N.C.. 2010. Ecological
and evolutionary interaction network exploration: addressing the
complexity of biological interactions in natural systems with
community genetics and statistics. Journal of integrated field science
7:17-25. fulltext PDF.

\item Whitham TG, CA Gehring, LM Evans, CJ LeRoy, RK Bangert, JA Schweitzer,
GJ Allan, RC Barbour, DG Fischer, BM Potts, and JK Bailey. 2010. A
Community and Ecosystem Genetics Approach for Conservation Biology and
Management. In: Molecular Approaches in Natural Resource Conservation
and Management (JA DeWoody, JW Bickham, C Michler, K Nichols, G
Rhodes, and K Woeste, eds.). Cambridge University Press :50-70.
 
\end{itemize}

\section{2009}
\begin{itemize}
\item Holeski, L.M., M.J.C. Kearsley, and T.G. Whitham. 2009. Separating
ontogenetic and environmental variation in resistance to herbivory in
cottonwood. Ecology 90:2969-2973. fulltext PDF.

\item Lojewski, N.R., D.G. Fischer, J.K. Bailey, J.A. Schweitzer,
T.G. Whitham and S.C. Hart. 2009. Genetic basis of aboveground
productivity in two native Populus species and their hybrids. Tree
Physiology 29:1133–1142. abstract. fulltext PDF.
 
\end{itemize}

\section{2008}
\begin{itemize}
\item Bailey, J.K.; Schweitzer, J.A.; Ubeda1, F.; Koricheva, J.; LeRoy,
C.J.; Madritch, M.D.; Rehill, B.J.; Bangert, R.K.; Fischer, D.G.;
Allan G.J. and T.G. Whitham. 2008. From genes to ecosystems: a
synthesis of the effects of plant genetic factors across levels of
organization.. Phil. Trans. R. Soc. B abstract. fulltext PDF.

\item Bangert R. K., E. V. Lonsdorf, G. M. Wimp, S. M. Shuster, D. Fischer,
J. A. Schweitzer, G. J. Allan, J. K. Bailey, and
T. G. Whitham. 2008. Genetic structure of a foundation species:
scaling community phenotypes from the individual to the
region. Heredity 100:121-131.

\item Evans,L.M., G.J. Allan, S.M. Shuster, S.A. Woolbright and
T.G. Whitham. 2008. Tree Hybridization and Genotypic Variation Drive
Cryptic Speciation of a Specialized Mite Herbivore. Evolution
62:3027-3040. abstract. fulltext PDF.

\item Schweitzer, J. A.; J. K. Bailey; D. G. Fischer; C. J. LeRoy;
E. V. Lonsdorf; T. G. Whitham and S. C. Hart. 2008. Soil
microorganism-plant interactions: A heritable relationship between
plant genotype and associated soil microorganisms. Ecology
89(3):773-781. abstract. fulltext PDF.

\item Schweitzer,J.A.; Madritch, M.D.; Bailey, J.K.; LeRoy, C.J.; Fischer,
D.G.; Rehill, B.J.; Lindroth, R.L.; Hagerman, A.E.; Wooley, S.C.;
Hart, S.C. and T.G. Whitham. 2008. From Genes to Ecosystems: The
Genetic Basis of Condensed Tannins and Their Role in Nutrient
Regulation in a Populus Model System. Ecosystems
11:1005–1020. abstract. fulltext PDF.

\item Whitham, T.G.; S.P. DiFazio; J.A. Schweitzer; S.M. Shuster;
G.J. Allan; J.K. Bailey and S.A. Woolbright. 2008. Extending Genomics
to Natural Communities and Ecosystems. Science
320:492-495. abstract. This article is also available online.
 
\item Woolbright, S. A.; S. P. DiFazio; T. Yin; G. D. Martinsen; X. Zhang;
G. J. Allan, T. G. Whitham; and P. Keim.. 2008. A dense linkage map of
hybrid cottonwood (Populus fremontii x P. angustifolia) contributes to
long-term ecological research and comparison mapping in a model forest
tree.. Heredity 100:59-70. abstract. fulltext PDF.This article is also
available online.

\item Yin, T.-M.; S.P. DiFazio, L.E. Gunter, X. Zhang, M.M. Sewell,
S.A. Woolbright, G.J. Allan, C.T. Kelleher, C.J. Douglas, M.-X. Wang,
G.A. Tuskan. 2008. Genome Structure and emerging evidence of an
incipient sex chromosome in Populus. Genome Research
18:422-430. abstract.
 
\end{itemize}

\section{2007}
\begin{itemize}
\item Bailey, J. K.; J. A. Schweitzer; B. J. Rehill; D. J. Irschick;
T. G. Whitham and R. L. Lindroth. 2007. Rapid shifts in the chemical
composition of aspen forests: An introduced herbivore as an agent of
natural selection. Biological Invasions 9:715-722. abstract. fulltext
PDF.

\item Bangert, RK, and TG Whitham. 2007. Genetic assembly rules and
community phenotypes. Evolutionary Ecology 21:549-560.

\item Fischer, D. G.; S. C. Hart; C. J. LeRoy and
T. G. Whitham. 2007. Variation in belowground carbon fluxes along a
Populus hybridization gradient. New Phytologist
176:415-425. abstract. fulltext PDF.
 
\item LeRoy, C. J.; T. G. Whitham; S. C. Wooley and
J. C. Marks. 2007. Within-species variation in foliar chemistry
influences aquatic leaf litter decomposition. Journal of the North
American Benthological Society 26(3):426-438. abstract. fulltext PDF.

\item Schweitzer JA, JK Bailey, RK Bangert, SC Hart and TG
Whitham. 2007. The role of plant genetic variation in determining
above- and belowground microbial communities. In: Microbial Ecology of
Arial Plant Surfaces, Bailey MJ, Lilley AK, Timms-Wilson TM and
Spencer-Phillips PTN (eds). CABI Publishing :109-119.

\item Wimp, G.M., S. Wooley, R. K. Bangert, W. P. Young, G. D. Martinsen,
P. Keim, R. L. Lindroth, and T. G. Whitham. 2007. Plant genetics
predicts intra-annual variation in phytochemistry and arthropod
community structure. Molecular Ecology
16:5057–5069. abstract. fulltext PDF.
 
\end{itemize}

\section{2006}
\begin{itemize}
\item Bailey, J. K. ; S. C. Wooley; R. L. Lindroth and
T. G. Whitham. 2006. Importance of species interactions to community
heritability: a genetic basis to trophic-level interactions. Ecology
Letters 9:78-85. abstract. fulltext PDF.

\item Bangert, RK, GJ Allan, RJ Turek, GM Wimp, N Meneses, GD Martinsen, P
Keim, and TG Whitham. 2006. From genes to geography: a genetic
similarity rule for arthropod community structure at multiple
geographic scales. Molecular Ecology 15:4215-4228.
 
\item Bangert, RK, RJ Turek, B Rehill, GM Wimp, JA Schweitzer, GJ Allan, JK
Bailey, GD Martinsen, P Keim, RL Lindroth, and TG Whitham. 2006. A
genetic similarity rule determines arthropod community
structure. Molecular Ecology 15:1379-1392.
 
\item Classen, A.T.; J. DeMarco; S.C. Hart; T. G. Whitham; N. S. Cobb and
G. W. Koch. 2006. Impacts of herbivorous insects on decomposer
communities during the early stages of primary succession in a
semi-arid woodland. Soil Biology and Biochemistry
38:972-982. abstract. fulltext PDF.
 
\item Fischer, D.G., S.C. Hart, B.J. Rehill, R.L. Lindroth, P. Keim and
T.G. Whitham. 2006. Do high-tannin leaves require more
roots?. Oecologia 149:668-675. abstract. fulltext PDF.
 
\item Gehring, C. A., Mueller, R. C. and T. G. Whitham. 2006. Environmental
and genetic effects on the formation of ectomycorrhizal and arbuscular
mycorrhizal associations in cottonwoods. Oecologia
149:158-164. fulltext PDF.
 
\item Gitlin, A.R., Sthultz, C.M., Bowker, M.A., Stumpf, S., Paxton, K.L.,
Kennedy, K., Munoz, A., Bailey, J.K., and
T.G. Whitham.. 2006. Mortality gradients within and among dominant
plant populations as barometers of ecosysetm change during extreme
drought.. Conservation Biology 20:1477-1486. abstract. fulltext PDF.
 
\item LeRoy, C. J.; T. G. Whitham; P. Keim and J. C. Marks. 2006. Plant
genes link forests and streams. Ecology 87:255-261. abstract. fulltext
PDF.
 
\item LeRoy, C.J. and J. C. Marks. 2006. Litter quality, stream
characteristics and litter diversity influence decomposition rates and
macroinvertebrates. Freshwater Biology 51:605-617. abstract. fulltext
PDF.
 
\item Muller, M.S.; McWilliams, S.R.; Podlesak, D.; Donaldson, J.R.;
Bothwell, H.M.; Lindroth, R.L.. 2006. Tri-trophic effects of plant
defenses: chickadees consume caterpillars based on host leaf
chemistry. Oikos 114:507-517. abstract. fulltext PDF.
 
\item Rehill, B.; T.G. Whitham; G.D. Martinsen; J.A. Schweitzer; J.K. Bailey
and R.L. Lindroth. 2006. Developmental trajectories in cottonwood
phytochemistry. Journal of Chemical Ecology
32:2269-2285. abstract. fulltext PDF.
 
\item Shuster, S. M.; E. V. Lonsdorf; G. M. Wimp; J. K. Bailey and
T. G. Whitham. 2006. Community heritability measures the evolutionary
consequences of indirect genetic effects on community
structure. Evolution 60:991-1003. abstract. fulltext PDF.
 
\item Whitham, T. G.; J. K. Bailey; J. A. Schweitzer; S. M. Shuster;
R. K. Bangert; C. J. LeRoy; E. V. Lonsdorf; G. J. Allan;
S. P. DiFazio; B. M. Potts; D. G. Fischer; C. A. Gehring;
R. L. Lindroth; J. C. Marks; S. C. Hart; G. M. Wimp and
S. C. Wooley. 2006. A framework for community and ecosystem genetics:
from genes to ecosystems. Nature Reviews Genetics 7:510 -
523. abstract. fulltext PDF.
 
\end{itemize}

\section{2005}
\begin{itemize}
\item Bangert, RK, RJ Turek, GD Martinsen, GM Wimp, JK Bailey, and TG
Whitham. 2005. Conservation of plant genetic diversity benefits
arthropod diversity. Conservation Biology 19:379-390.
 
\item Classen, A. T.; S. C. Hart; T. G. Whitman; N. S. Cobb and
G. W. Koch. 2005. Insect infestations linked to shifts in
microclimate: Important climate change implications. Soil Science
Society of America 69:2049-2057. abstract. fulltext PDF.
 
\item Cox, G.; D.G. Fischer; S.C. Hart; and T.G. Whitham. 2005. Nonresponse
of native cottonwood trees to water additions during summer
drought. Western North American Naturalist
65:175-185. abstract. fulltext PDF.
 
\item Rehill, B.; A. Clauss; L. Wieczorek; T. Whitham and
R. Lindroth. 2005. Foliar phenolic glycosides from Populus fremontii,
Populus angustifolia, and their hybrids. Biochemical Systematics and
Ecology 33:125-131. abstract. fulltext PDF.
 
\item Schweitzer, J. A.; J. K. Bailey and S. C. Hart. 2005. Synergistic
effects of plant genotype and herbivory alter litter decomposition and
nutrient flux. Ecology
 
\item Schweitzer, J. A.; J. K. Bailey; S. C. Hart; G. M. Wimp; S. K. Chapman
and T. G. Whitham. 2005. The interaction of plant genotype and
herbivory decelerate leaf litter decomposition and alter nutrient
dynamics. Oikos 110:133-145. abstract. fulltext PDF.
 
\item Shuster, T. D.; N. S. Cobb; T. G. Whitham and
S. C. Hart. 2005. Relative importance of environmental stress and
herbivory in reducing litter fall in a semiarid woodland. Ecosystems
8:62-72. abstract. fulltext PDF.
 
\item Whitham, T. G.; E. Lonsdorf; J. A. Schweitzer; J. K. Bailey;
D. G. Fischer; S. M. Shuster; R. L. Lindroth; S. C. Hart; G. J. Allan;
C. A. Gehring; P. Keim; B. M. Potts; J. Marks; B. J. Rehill;
S. P. DiFazio; C. J. LeRoy; G. M. Wimp; and S. Woolbright. 2005. ``All
effects of a gene on the world'': extended phenotypes, feedbacks, and
multi-level selection. Ecoscience 12:5-7. abstract. fulltext PDF.
 
\item Wimp, G. M.; G. D. Martinsen; K. D. Floate; R. K. Bangert and
T. G. Whitham. 2005. Plant genetic determinants of arthropod community
structure and diversity. Evolution 59:61-69. abstract. fulltext PDF.
 
\end{itemize}

\section{2004}
\begin{itemize}
\item Bailey, J. K.; J. A. Schweitzer; B. J. Rehill; R. L. Lindroth;
G. D. Martinsen and T. G. Whitham. 2004. Beavers as molecular
geneticists: a genetic basis to the foraging of an ecosystem
engineer. Ecology 85:603-608. abstract. fulltext PDF.
 
\item Bailey, J. K.; R. K. Bangert; J. A. Schweitzer; R. T. Trotter;
S. M. Shuster and T. G. Whitham. 2004. Fractal geometry is heritable
in trees. Evolution 58:2100-2102. abstract. fulltext PDF.
 
\item Ferrier, S. M. and P. W. Price. 2004. Oviposition Preference and
Larval Performance of a Rare Bud-Galling Sawfly (Hymenoptera:
Tenthredinidae) on Willow in Northern Arizona. Environmental
Entomology 33:700-708. abstract. fulltext PDF.
 
\item Fischer, D. G.; S. C. Hart; T. G. Whitham; G. D. Martinsen and
P. Keim. 2004. Ecosystem implications of genetic variation in
water-use of a dominant riparian tree. Oecologia
139:288-297. abstract. fulltext PDF.
 
\item Flaccus, K.; J. Vlieg; J. C. Marks and C. J. LeRoy. 2004. Restoring
Fossil Creek. The Science Teacher 71:36-40. fulltext PDF.
 

\item Floate, K. D. 2004. Extent and patterns of hybridization among the
three species of Populus that constitute the riparian forest of
southern Alberta, Canada. Canadian Journal of Botany 82:253-264.
 
\item Osier, T. L. and R. L. Lindroth. 2004. Long-term effects of
defoliation on quaking aspen in relation to genotype and nutrient
availability: plant growth, phytochemistry and insect
performance. Oecologia 139:55-65. abstract.
 
\item Price, P.W.; T. Ohgushi; H. Roininen; M. Ishihara; T.P. Craig;
J. Tahvanainen and S.M. Ferrier. 2004. Release of phylogenetic
constraints through low resource heterogeneity: the case of
gall-inducing sawflies. Ecological Entomology
29:467-481. abstract. fulltext PDF.
 
\item Schweitzer, J. A.; J. K. Bailey; B. J. Rehill; G. D. Martinsen;
S. C. Hart; R. L. Lindroth; P. Keim and
T. G. Whitham. 2004. Genetically based trait in a dominant tree
affects ecosystem processes. Ecology Letters
7:127-134. abstract. fulltext PDF.
 
\item Wimp, G. M.; W. P. Young; S. A. Woolbright; G. D. Martinsen; P. Keim
and T. G. Whitham. 2004. Conserving plant genetic diversity for
dependent animal communities. Ecology Letters
7:776-780. abstract. fulltext PDF.
 
\end{itemize}

\section{2003}
\begin{itemize}
\item Bailey, J. K. and T. G. Whitham. 2003. Interactions among elk, aspen,
galling sawflies and insectivorous birds. Oikos
101:127-134. abstract. fulltext PDF.
 
\item Chapman, S. K.; S. C. Hart; N. S. Cobb; T. G. Whitham and
G. W. Koch. 2003. Insect herbivory increases litter quality and
decomposition: an extension of the acceleration hypothesis. Ecology
84:2867-2876. abstract. fulltext PDF.
 
\item Lawrence, R.; B. M. Potts and T. G. Whitham. 2003. Relative importance
of plant ontogeny, host genetic variation, and leaf age for a common
herbivore. Ecology 84:1171-1178. abstract. fulltext PDF.
 
\item McIntyre, P. J. and T. G. Whitham. 2003. Plant genotype affects
long-term herbivore population dynamics and extinction: conservation
implications. Ecology 84:311-322. abstract. fulltext PDF.
 
\item Whitham, T. G.; W. P. Young; G. D. Martinsen; C. A. Gehring;
J. A. Schweitzer; S. M. Shuster; G. M. Wimp; D. G. Fischer;
J. K. Bailey; R. L. Lindroth; S. Woolbright and
C. R. Kuske. 2003. Community and ecosystem genetics: A consequence of
the extended phenotype. Ecology 84:559-573. abstract. fulltext PDF.
 
\end{itemize}

\section{2002}
\begin{itemize}
\item Bailey, J. K. and T. G. Whitham. 2002. Interactions among fire, aspen,
and elk affect insect diversity: reversal of a community
response. Ecology 83:1701-1712. abstract. fulltext PDF.
 
\item Lindroth, R. L.; S. A. Wood and B. J. Kopper. 2002. Response of
quaking aspen genotypes to enriched CO2: foliar chemistry and tussock
moth performance. Agricultural and Forest Entomology
4:315-323. abstract.
 
\item Schweitzer, J. A.; G. D. Martinsen and T. G. Whitham. 2002. Cottonwood
hybrids gain fitness traits of both parents: a mechanism for their
long-term persistence?. American Journal of Botany
89:981-990. abstract. fulltext PDF.
 
\end{itemize}

\section{2001}
\begin{itemize}
\item Bailey, J. K.; J. A. Schweitzer and T. G. Whitham. 2001. Salt cedar
negatively affects biodiversity of aquatic
macroinvertebrates. Wetlands 21:442-447. abstract. fulltext PDF.
 
\item Brown, J. H.; T. G. Whitham; S. K. Morgan Ernest and
C. A. Gehring. 2001. Complex species interactions and the dynamics of
ecological systems: long-term experiments. Science
293:643-650. abstract. fulltext PDF.
 
\item Lindroth, R. L.; B. J. Kopper; W. F. J. Parsons; J. G. Bockheim;
D. F. Karnosky; G. R. Hendrey; K. S. Pregitzer; J. G. Isebrands and
J. Sober. 2001. Consequences of elevated carbon dioxide and ozone for
foliar chemical composition and dynamics in trembling aspen (Populus
tremuloides) and paper birch (Betula papyrifera). Environmental
Pollution 115:395-404. abstract.
 
\item Martinsen, G. D.; T. G. Whitham; R. J. Turek and P. Keim. 2001. Hybrid
populations selectively filter gene introgression between
species. Evolution 55:1325-1335. abstract. fulltext PDF.
 
\item Osier, T. L. and R. L. Lindroth. 2001. Effects of genotype, nutrient
availability, and defoliation on aspen phytochemistry and insect
performance. Journal of Chemical Ecology 27:1289-1313. abstract.
 
\item Wimp, G. M. and T. G. Whitham. 2001. Biodiversity consequences of
predation and host plant hybridization on an aphid-ant
mutualism. Ecology 82:440-452. abstract. fulltext PDF.
 
\end{itemize}

\section{2000}
\begin{itemize}
\item Driebe, E. M. and T. G. Whitham. 2000. Cottonwood hybridization
affects tannin and nitrogen content of leaf litter and alters
decomposition. Oecologia 123:99-107. abstract. fulltext PDF.
 
\item Dungey, H. S.; B. M. Potts; T. G. Whitham and H. F. Li. 2000. Plant
genetics affects arthropod community richness and composition:
evidence from a synthetic eucalypt hybrid population. Evolution
54:1938-1946. abstract. fulltext PDF.
 
\item Martinsen, G. D.; K. D. Floate; A. M. Waltz; G. M. Wimp and
T. G. Whitham. 2000. Positive interactions between leafrollers and
other arthropods enhance biodiversity on hybrid cottonwoods. Oecologia
123:82-89. abstract. fulltext PDF.
 
\item White, J. A. and T. G. Whitham. 2000. Associational susceptibility of
cottonwood to a box elder herbivore. Ecology
81:1795-1803. abstract. fulltext PDF.
 
\end{itemize}

\section{1999}
\begin{itemize}
\item Lindroth, R. L.; S. Y. Hwang and T. L. Osier. 1999. Phytochemical
variation in quaking aspen: Effects on gypsy moth susceptibility to
nuclear polyhedrosis virus. Journal of Chemical Ecology
25:1331-1343. abstract.
 
\item Whitham, T. G.; G. D. Martinsen; K. D. Floate; H. S. Dungey;
B. M. Potts and P. Keim. 1999. Plant hybrid zones affect biodiversity:
tools for a genetic-based understanding of community
structure. Ecology 80:416-428. abstract. fulltext PDF.
 
\end{itemize}

\section{1998}
\begin{itemize}
\item Gehring, C. A.; T. C. Theimer; T. G. Whitham and
P. Keim. 1998. Ectomycorrhizal fungal community structure of pinyon
pines growing in two environmental extremes. Ecology
79:1562-1572. abstract. fulltext PDF.
 
\item Kearsley, M. J. and T. G. Whitham. 1998. The developmental stream of
cottonwoods affects ramet growth and resistance to galling
aphids. Ecology 79:178-191. abstract. fulltext PDF.
 
\item Lindroth, R. L. and K. K. Kinney. 1998. Consequences of enriched
atmospheric CO2 and defoliation for foliar chemistry and gypsy moth
performance. Journal of Chemical Ecology 24:1677-1696. abstract.
 
\item Martinsen, G. D.; E. M. Driebe and T. G. Whitham. 1998. Indirect
interactions mediated by changing plant chemistry: beaver browsing
benefits beetles. Ecology 79:192-200. abstract. fulltext PDF.
 
\end{itemize}

\section{1997}
\begin{itemize}
\item Cobb, N. S.; S. Mopper; C. A. Gehring; M. Caouette;
  K. M. Christensen and T. G. Whitham. 1997. Increased moth herbivory
  associated with environmental stress of pinyon pine at local and
  regional levels. Oecologia 109:389-397. abstract. fulltext PDF.
 
\item Floate, K. D.; G. D. Martinsen and T. G. Whitham. 1997. Cottonwood
hybrid zones as centres of abundance for gall aphids in western North
America: importance of relative habitat size. Journal of Animal
Ecology 66:179-188. abstract. fulltext PDF.
 
\item Gehring, C. A.; N. S. Cobb and T. G. Whitham. 1997. Three-way
interactions among ectomycorrhizal mutualists, scale insects, and
resistant and susceptible pinyon pines. American Naturalist
149:824-841. abstract. fulltext PDF.
 
\item Larson, K. C. and T. G. Whitham. 1997. Competition between gall aphids
and natural plant sinks: plant architecture affects resistance to
galling. Oecologia 109:575-582. abstract. fulltext PDF.
 
\item Lindroth, R. L.; K. A. Klein; J. D. C. Hemminen chemistry: Effects on
gypsy moth performance and susceptibility to virus. Global Change
Biology 3:279-289. abstract.
 
\item Waltz, A.M. and T.G. Whitham. 1997. Plant development affects
arthropod communities: opposing impacts of species removal. Ecology
78:2133-2144. abstract. fulltext PDF.
 
\end{itemize}

\section{1996}
\begin{itemize}
\item Dickson, L. L. and T. G. Whitham. 1996. Genetically-based plant
resistance traits affect arthropods, fungi, and birds. Oecologia
106:400-406. abstract. fulltext PDF.
 
\item Fay, P. A.; R. W. Preszler and T. G. Whitham. 1996. The functional
resource of a gall-forming adelgid. Oecologia
105:199-204. abstract. fulltext PDF.
 
\item Floate, K. D.; G. W. Fernandes and J. A. Nilsson. 1996. Distinguishing
intrapopulational categories of plants by their insect faunas: galls
on rabbitbrush. Oecologia 105:221-229.
 \end{itemize}

\section{1995}
\begin{itemize}
\item Keim. 1995. Herbivory and tree mortality across a pinyon pine hybrid
zone. Oecologia 101:29-36. abstract. fulltext PDF.
 
\item Floate, K. D. and T. G. Whitham. 1995. Insects as traits in plant
systematics: their use in discriminating between hybrid
cottonwoods. Canadian Journal of Botany 73:1-13. abstract.
 
\item Gehring, C. A. and T. G. Whitham. 1995. Duration of herbivore removal
and environmental stress affect the ectomycorrhizae of pinyon
pines. Ecology 76:2118-2123. abstract. fulltext PDF.
 
\end{itemize}

\section{1994}
\begin{itemize}
\item Floate, K. D. and T. G. Whitham. 1994. Aphid-ant interaction
reduces. Morrow and B. M. Potts. 1994. Plant hybrid zones as centers
of biodiversity: the herbivore community of two endemic Tasmanian
eucalypts. Oecologia 97:481-490. abstract. fulltext PDF.
 
\end{itemize}

\section{1993}
\begin{itemize}
\item Christensen K. M. and T. G. Whitham. 1993. Impact of insect herbivores
on competition between birds and mammals for pinyon pine
seeds. Ecology 74:2270-2278. abstract. fulltext PDF.
 
\item Floate K. D. and T. G. Whitham. 1993. The hybrid bridge hypothesis:
host shifting via plant hybrid swarms. American Naturalist
141:651-662. fulltext PDF.
 
\item Floate, K. D.; M. J. C. Kearsley and T. G. Whitham. 1993. Elevated
herbivory in plant hybrid zones: Chrysomela confluens, Populus and
phenological sinks. Ecology 74:2056-2065. abstract. fulltext PDF.

\item Lindroth, R. L.; P. B. Reich; M. G. Tjoelker; J. C. Volin and
J. Oleksyn. 1993. Light environment alters response to ozone stress in
seedlings of Acer saccharum Marsh. and hybrid Populus
L. III. Consequences for performance of gypsy moth. New Phytologist
124:647-651. abstract.
\end{itemize}

\section{1992}
\begin{itemize}
\item Gehring, C. A. and T. G. Whitham. 1992. Reduced mycorrhizae on
Juniperus monosperma with mistletoe: the influence of environmental
stress and tree gender on a plant parasite and a plant-fungal
mutualism. Oecologia 89:298-303. abstract. fulltext PDF.
 
\item Kearsley, M. J. C. and T. G. Whitham. 1992. Guns and butter: A no cost
defense against predation for Chrysomela confluens. Oecologia
92:556-562. abstract. fulltext PDF.
 
\item Mopper, S. and T. G. Whitham. 1992. The plant stress paradox: effects
on pinyon sawfly sex ratios and fecundity. Ecology
73:515-525. abstract. fulltext PDF.
 
\end{itemize}

\section{1991}
\begin{itemize}
\item Christensen, K. M. and T. G. Whitham. 1991. Indirect herbivore
mediation of avian seed dispersal in pinyon pine. Ecology
72:534-542. abstract. fmycorrhizal mutualism in insect-susceptible
pinyon pine. Nature 353:556-557. abstract. fulltext PDF.
 
\item Larson, K. C. and T. G. Whitham. 1991. Manipulation of food resources
by a gall-forming aphid: the physiology of sink-source
interactions. Oecologia 88:15-21. abstract. fulltext PDF.
 
\item Lindroth, R. L.. 1991. Biochemical ecology of aspen-Lepidoptera
interactions. Journal of the Kansas Entomological Society
64:372-380. abstract.
 
\item Lindroth, R. L. and A. V. Weisbrod. 1991. Genetic variation in
response of the gypsy moth to aspen phenolic glycosides. Biochemical
Systematics and Ecology 19:97-103. abstract.
 
\item Lindroth, R. L. and M. S. Bloomer. 1991. Biochemical ecology of the
forest tent caterpillar: Responses to dietary protein and phenolic
glyract.
 
\item Mopper, S.; J. B. Mitton; T. G. Whitham; N. S. Cobb and
K. M. Christensen. 1991. Genetic differentiation and heterozygosity in
pinyon pine associated with resistance to herbivory and environmental
stress. Evolution 45:989-999. abstract. fulltext PDF.
 
\end{itemize}

\section{1990}
\begin{itemize}
\item Fay, P. A. and T. G. Whitham. 1990. Within-plant distribution of a
galling adelgid (Homoptera: Adelgidae): the consequences of
conflicting survivorship, growth, and reproduction. Ecological
Entomology 15:245-254. abstract.
 
\item Lindroth, R. L. and J. D. C. Hemming. 1990. Responses of the gypsy
moth (Lepidoptera: Lymantriidae) to tremulacin, an aspen phenolic
glycoside. Environmental Entomology 19:842-847. abstract.
 
\item Lindroth, a shortgrass prairie community. Oecologia
83:132-138. abstract. fulltext PDF.
 
\item Mopper, S.; T. G. Whitham and P. W. Price. 1990. Plant phenotype and
interspecific competition between insects determine sawfly performance
and density. Ecology 71:2135-2144. abstract. fulltext PDF.
 
\item Moran, N. A. and T. G. Whitham. 1990. Interspecific competition
between root-feeding and leaf-galling aphids mediated by host-plant
resistance. Ecology 71:1050-1058. abstract. fulltext PDF.
 
\item Moran, N. A. and T. G. Whitham. 1990. Differential colonization of
resistant and susceptible host plants: Pemphigus and Populus. Ecology
71:1059-1067. abstract. fulltext PDF.
 
\end{itemize}

\section{1989}
\begin{itemize}
\item Kearsley, M. J. and T. G. Whitham. 1989. Developmental changes in
resistance to herbivory: implications for individuals and
populations. Ecology 70:422-434. abstract. fulltext PDF.

\item Maschinski, J. and T. G. Whitham. 1989. The continuum of plant
responses to herbivory: the influence of plant association, nutrient
availability, and timing. American Naturalist
134:1-19. abstract. fulltext PDF.
 
\item Whitham, T. G.. 1989. Plant hybrid zones as sinks for pests. Science
244:1490-1493. fulltext PDF.
 
\end{itemize}

\section{1988}
\begin{itemize}
\item Moran, N. A. and T. G. Whitham. 1988. Population fluctuations in
complex life cycles: an example from Pemphigus aphids. Ecology
69:1214-1218. abstract. fulltext PDF.
 
\end{itemize}

\section{1987}
\begin{itemize}
\item Paige, K. N. and T. G. Whitham. 1987. Flexible life history traits:
Shifts by scarlet gilia in response to pollinator abundance. Ecology
68:1691-1695. abstract. fulltext PDF.

\item Paige, K. N. and T. G. Whitham. 1987. Overcompensation in response to
mammalian herbivory: The advantage of being eaten. American Naturalist
129:407-416. abstract. fulltext PDF. 

\item Whitham, T. G.. 1987. Evolution of territoriality by herbivores in
response to host plant defenses. American Zoologist 27:359-369. abstract.
\end{itemize}

\section{1986}
\begin{itemize}
\item Whitham, T. G.. 1986. Costs and benefits of territoriality: behavioral
and reproductive release by competing aphids. Ecology
67:139-147. abstract. fulltext PDF.
 
\item Williams, A. G. and T. G. Whitham. 1986. Premature leaf abcission: an
induced plant defense against gall aphids. Ecology
67:1619-1627. fulltext PDF.
 
\end{itemize}

\section{1985}
\begin{itemize}
\item Whitham, T. G. and S. Mopper. 1985. Chronic herbivory: impacts on
architecture and sex expression of pinyon pine. Science
228:1089-1090. abstract. fulltext PDF.
\end{itemize}

\section{1980}
\begin{itemize}
\item Whitham, T. G.. 1980. The theory of habitat selection: examined and
extended using Phemphigus aphids. The American Naturalist 115:4
\end{itemize}

\section{1978}
\begin{itemize}
\item Whitham, T. G.. 1978. Habitat selection by Pemphigus aphids in a
response to resource limitation and competition. Ecology
59:1164-1176. fulltext PDF.
\end{itemize}

%% %%Activate for bibtex vibliography
%% \cite{goossens93}
%% \bibliographystyle{plain}
%% \bibliography{/Users/Aeolus/Documents/bibtex/biblib}


\end{document}  


