\documentclass[12pt]{article}
\usepackage{color}
\usepackage{cite}
\usepackage{geometry}                % See geometry.pdf to learn the layout options. There are lots.
%\usepackage{pdflscape}        %single page landscape
                                %mode \begin{landscape} \end{landscape}
\geometry{letterpaper}                   % ... or a4paper or a5paper or ... 
%\usepackage[parfill]{parskip}    % Activate to begin paragraphs with an empty line rather than an indent
\usepackage{graphicx}
\usepackage{amssymb}
\usepackage{/Library/Frameworks/R.framework/Resources/share/texmf/Sweave}
\newcommand{\etal}{\textit{et al.}}
\usepackage{hyperref}  %\hyperref[label_name]{''link text''}
                       %\hyperlink{label}{anchor caption}
                       %\hypertarget{label}{link caption}
\linespread{1.0}

\title{Dissertation Notes}
\author{M.K. Lau}
%\date{}                                           % Activate to display a given date or no date

\begin{document}
\maketitle

%Tom has suggested possibly doing your orals this spring, it would
%have to be before May 12. What is the benefit of doing them then and
%what would you need to do them by then?

%It's more important to get out pubs then to finish, unless you need
%to advance to candidacy, e.g. DDIG.

\section*{Dissertation Requirements Timeline}

\begin{itemize}
\item \textbf{April 2012} Assessment, prospectus review and course plan approval
  \subitem Bio form 5 - PhD Program Form
  \subitem Bio form 10 - Progress/Funding Assessment
  \subitem Bio form 13 - Teaching requirement documentation
  \subitem Bio form 14 - Scientific paper presentation documentation
%\item \textbf{Oct 2012} Grant and written exam approval
\item \textbf{Feb 2013} Prospectus defense and oral exam
  \subitem Bio form 7 - Written exam results
  \subitem Bio form 10 - Progress/Funding Assessment
  \subitem Bio form 8 - Oral exam results REPORT
  \subitem Bio form 8.11a - Oral exam assessment/questionaire
  \subitem Bio form 9 - Prospectus approval form
  \subitem Bio form 11 - Oral exam results
  \subitem Bio form 10 - Progress/Funding Assessment
\item \textbf{May 2014} Final dissertation defense
  \subitem Dissertation draft
  \subitem Dissertation Defense Scheduling Form
  \subitem Graduate college final oral exam form (can only be accessed
  by advisor)
\end{itemize}

\section*{Committee}
\subsection{Requirements}
\begin{enumerate}
\item At least four members
\item At least one external member
\item Majority from within the department (must have voting status;
  this includes emeritus faculty)
\item 
\end{enumerate}

\section{The Evolution of Interaction Networks}

\subsection{Stability of Communities and Evolution}
MacArthur 1955

\begin{enumerate}
\item Assuming that:
  \begin{enumerate}
  \item the ammount of energy entering the community does not vary
    with time
  \item the length of time that energy is retained by a species
    doesnt' change with time
  \item the population of each species varies directly with available
    food energy
  \end{enumerate}
\item These assumptions imply that the population of each species
  tends to a specific constant (example Linderman 1942)
\item $p_{ij}$ = the proportion of energy transfered from species $i$
  to $j$
\item As long as all energy transfers are shown/quantified, then
  $\sum_j p_{ij} = 1$
\item \quote{``This equation shows that the food web considered as an
  energy transformer is what is known in probability theory as a
  Markov chain (Feller 1950).''}
\item Given the assumptions, the species' populations should tend
  toward a constant.
\item However, since populations fluctuate in nature, at least one of
  the assumptions must not hold.
\item Community Stability = the constancy of species' abundances over
  time
\item Stability can arise in two ways:
  \begin{enumerate}
  \item Patterns of interactions among species
  \item Properties intrinsic to each species
  \end{enumerate}
\item Stability ($S$) increases with the number of pathways ($p_i$), as $S =
  -\sum p_i log(p_i)$
\item There are several interesting properties of stability:
  \begin{enumerate}
  \item stability increases with the total number of links
  \item stability will increase with richness (given the number of
    prey per species remaincs constant)
  \item Given 1 and 2, stability can be achieved by a large number of
    species with a restricted diet or a few species with a large
    number of prey
  \item Maximum stability for $m$ species is achieved by $m$ trophic
    levels with every species eating all species in lower trophic
    levels
  \item Minimum stability would arise with one species eating all
    others in a single lower trophic level
  \end{enumerate}
\end{enumerate}


\section{How does genetic variation influence ecological network
  structure?}
\begin{itemize}
\item Does genetic variation within a foundation species lead to
  modularity in co-occurrence patterns of associated species?
\item Approach:
  \begin{enumerate}
  \item Use Lonsdorf's simulation to generate communities
  \item Model communities as bipartite graphs using relative
    abundances
  \item Generate structural statistics (including modularity and
    nestedness)
  \item Look for relationship with genetic variability
  \item Add layers to simulation, such as interactions among
    arthropods (e.g. intransitivity)
  \end{enumerate}
\end{itemize}

\section{How does genetic variability contribute to the structure of
  plant-herbivore (and predator) networks?}
\begin{itemize}
\item Does the modular structure of plant-herbivore interaction
  networks arise from crosstype or genotype variability?
\item Approach:
  \begin{enumerate}
  \item Gina's, Art's, Sharon's and Dave's arthropod data
  \item Reduce to just known herbivores (and maybe omnivores) and
    analyze for modularity
  \item Use herbivore and predator co-occurrences to generate
    herbivore-predator (and maybe parasite) networks and analyze for
    network structural similarity among crosstypes/genotypes
  \end{enumerate}
\end{itemize}

\section{How does genotypic variability influence the structure of
  lichen species interactions?}
\begin{itemize}
\item How do lichen co-occurrence patterns vary among genotypes?
\item Approach:
  \begin{enumerate}
  \item Collect co-occurrence data for bark lichen associated with
    cottonwood individuals of known genotype
  \item Model co-occurrence patterns using standard co-occurrence
    methods and network modeling algorithms
  \item Test for the effect or similarity of genotypes with respect to
    lichen co-occurrence structure
  \item Analyze the structural patterns that are changing among
    genotypes
  \item Analyze the individual species responses
  \end{enumerate}
\end{itemize}

\section{How does genotypic variation influence the structure of
  modifier-inquiline interactions?}
\begin{itemize}
\item How does host genotype directly and indirectly influence the
  interaction between leaf-modifiers and inquilines associated with
  their modification structures?
\item Approach:
  \begin{enumerate}
  \item Survey the abundance of leaf-modifiers in the wild along a
    hybridizing system
  \item Survey leaf-modifiers and their inquilines in the common garden
  \item Create artificial leaf-modifications that mimic the
    leaf-modifier species (i.e. paper clip method used in Martinsen et
    al. 2000)
  \item Survey leaf modifications for inquilines
  \item Build networks for each tree using the inquiline abundance
    data (possibly separate out known herbivores from predators)
  \item Analyze networks for structural similarity among genotypes
  \item Analyze for individual species responses (locations in
    networks)
  \end{enumerate}
\end{itemize}

\section{How does plant facilitation influence the co-occurrence patterns
of associated plant species?}

\begin{itemize}
\item Are there phylogenetic patterns in the effect of nurse plant on
  associated species co-occurrences?
\item Approach:
  \begin{enumerate}
  \item Cleanup Alpine Pals' global community dataset
  \item Get phylo-stats from Brad
  \item Model co-occurrence networks using distance based method
  \item Look for correlation between phylo-stats and co-occurrence
    network structure
  \item Explore to find the source of the phylogenetic patterns
  \end{enumerate}
\end{itemize}

\begin{itemize}
\item How does intraspecific variation influence inter-plant interactions?
\item Approach:
  \begin{enumerate}
  \item Get Lamit's plant data
  \item Model as a bipartite graph
  \item Analyze for modularity
  \end{enumerate}
\end{itemize}


%% \section{23July2009}
%% \begin{description}
%% \item{Are ecological interaction networks constrained by evolutionary processes?}
%%   \begin{description}
%%   \item{Multi-level selection theory suggests that organisms can
%%     evolve due to evolutionary forces from other organisms in the
%%     community. They do not even have to directly interact in order to
%%     be linked evolutionarily, because indirect interactions can also
%%     produce significant evolutionary pressures. Selection at the group
%%   }
%%   \end{description}
  
%% \end{description}

%% \section{19July2009}
%% \begin{description}
%% \item{At any level, biological organization arises via a
%%   \textit{process} of development. It does not arise spontaneously
%%   formed in its entirety. For example, individual organisms are formed
%%   by a process of cellular replication, growth and
%%   differentiation. The outcome is determined in part by genes,
%%   environment and chance. Genes are not a blue-print, specifying the
%%   exact details of an organism, but more like an algorithm comprised
%%   of a set simple instructions (i.e. information on how to respond to
%%   a given environmental condition). How this information is passed on
%%   is inheritance, and the results of changes to this information is
%%   evolution.}

%% \item{"If genes change in evolution, the environment of the organism
%%   will change too." - Richard Lewontin (1992)}

%% \item{Biological evolution is a response of problem solving organisms
%%   to changing environments.}

%% \end{description}

%% \section{13 July 2009}
%% \subsection{Questions}
%% \begin{enumerate}
%% \item{Is the path from genes to communities unilateral (i.e. genes
%%   -$>$ classical traits -$>$ species abundances), interactive
%%   (i.e. genes -$>$ classical traits -$>$ species interactions -$>$
%%   species abundances) or both?}
%% \item{How do you quantify the Minimum Viable Interacting Population
%%   (MVIP)?}
%% \end{enumerate}

%% \subsection{Projects}
%% \begin{enumerate}
%% \item{Develop an improved community simulator that incorporates:}
%% 	\begin{enumerate}
%% 	\item{Communities (i.e. multiple species that can interact)}
%% 	\item{Multiple loci (to model epistasis)}
%% 	\item{More Traits}
%% 	\item{More alleles}
%% 	\item{Dominance}
%% 	\end{enumerate}
%% \end{enumerate}

%% \section{Summary -- A framework for community and ecosystem genetics:
%%   from genes to ecosystems}

%% Since heritable, genetically-determined traits have predictable
%% effects on community structure and ecosystem processes, principles of
%% quantitative population and quantitative genetics can be applied to
%% studies of communities and ecosystems within an evolutionary
%% framework.

%% HOT QUESTION: What are the effects of climate change and transgenic
%% organisms on entire communities and/or ecosystems?

%% \begin{itemize}
%% \item{Only recently has broader community and ecosystem evolution
%%   beyond the evolution and ecology of two-species interactions
%%   (Ehrlich and Raven)}
%% \item{Although, multi-level selection theory has advanced, it has not
%%   been incorporated into models of community organization and
%%   ecosystem dynamics}
%% \item{Overwhelming species numbers and the complexity of species
%%   interactions are a major deterrent to advancing community and
%%   ecosystem genetics}
%% \item{Studying poplars as a model taxon has allowed three conceptual
%%   advances:}

%%   \begin{enumerate}
%%   \item{Genetic analysis of foundation species can tell us much about
%%     an ecosystem}
%%   \item{The predictable effects of genes in foundation species extend
%%     to higher levels, which can be quantified (i.e. higher level
%%     phenotypes)}
%%   \item{The community and ecosystem heritability of these phenotypes
%%     can be quantified using standard community statistical tools,
%%     which can then be analyzed using established population and
%%     quantitative genetics analyses}
%%   \end{enumerate}
%% \item{HOT QUESTION: Do higher level phenotypes feed back to affect
%%   fitness of the individual expressing the trait?}
%% \item{HOT QUESTION: Do communities and ecosystems evolve?}
%% \item{HOT QUESTION: Will higher level phenotypes of genetically
%%   modified organisms negatively affect the environment?}
%% \item{HOT QUESTION: Is it important to conserve genetic variation in
%%   foundation species that are common?}
%% \item{The C-wood model of higher-level evolution does not require or
%%   imply higher-level fitness or evolution analogous to population
%%   level fitness and evolution. However, because individual heritable
%%   phenotypes can influence the fitness and phenotypes of other
%%   species, selection occurs within a community context.}
%% \end{itemize}

%% \paragraph{From community genetics perspective, the community
%%   associated with a cottonwood forest is like a blanket of inter-woven
%%   threads formed by species interactions. Localized interactions
%%   between individual trees other trees and the organisms on those
%%   trees percolate through the blanket. The blanket itself changes in
%%   form as a result of these interactions as well as the effects of
%%   abiotic environmental factors (e.g. climate, nutrient flow,
%%   sun-light).}

%% \paragraph{The term foundation species is mentioned 37 times, with
%%   only one reference.}

%% \section{Perspective}

%% \begin{enumerate}
%% \item{The behavior (i.e. future states) of complex systems cannot be
%%   deduced from the behaviors of the component parts of the system.}
%% \item{Investigations of species interactions in natural systems and
%%   food-webs have shown that communities are complex systems in that
%%   the ecological interactions amongst species result in emergent
%%   properties that are not wholly predictable from pairwise
%%   interactions. To improve our ability to understand and potentially
%%   predict community dynamics, we need an analytical approach that can
%%   use information about the entire system.}
%% \item{Network Theory provides a useful approach for studying
%%   ecological systems with many interacting components, such as
%%   communities.}
%% \end{enumerate}

%% \section{Network Theory and Community Evolution}
%% Both theoretical and empirical studies have shown that genetically
%% based, phenotypic variation amongst individuals in a population can
%% result in variation in their associated communities (Shuster et
%% al. 2006, Whitham et al. 2006 and Whitham et al. 2008). This provides
%% a very useful framework for using genetic information of a species to
%% predict variation in the community, when applied to a species that has
%% a strong influence within an ecosystem (i.e. a foundation species).

%% \paragraph{Questions:}
%% \begin{itemize}
%% \item{What is a foundation species?}
%% \item{More specifically, how do we distinguish a foundation species
%%   from other species?}
%% \item{Is it a species who's removal will cause large fluctuations?}
%% \end{itemize}
%% The current methodology uses an ordination based approach that
%% summarizes the variation in species abundances, usually with
%% non-metric multidimensional scaling (NMS). This approach, although
%% good for quantifying the important components of the community that
%% are responding to variation in genetics, it does not incorporate any
%% information about the structure of the interactions amongst
%% species. We know from studies of ecological networks that the
%% structure of the network (i.e. how interactions amongst species are
%% distributed) can have large consequences for the behavior of the
%% network (reviewed in Montoya et al. 2005). It follows that
%% incorporating network information into the theory, models or analyses
%% of communities will improve our understanding of the process and
%% outcomes of community evolution (see Fig. 1). 
%% \paragraph{Questions:}

%% \begin{itemize}
%% \item{Can we determine foundation species by analyzing the
%%   species-interaction network?}
%% \item{How can we incorporate network theory into the current theory,
%%   models and analyses of community evolution?}
%% \item{Can we infer network structure from covariance or correlation
%%   matrixes of species abundances?}
%% \end{itemize}

%% \begin{center}
%%   \begin{figure}[h]
    
%%     %<<fig=true,echo=false,width=6,height=6>>=
%%     %library(sna)
%%     %ns = 3000 #number of species 
%%     %no = 100 #number of observations
%%     %svr = c(1,1000) #species vector range
%%     %spv = c(round(runif(ns,svr[1],svr[2]))) #shape parameter vector
%%     %sam = array(NA,c(no,ns)) #species abundance matrix
    
%%     %for (i in 1:ns){sam[,i]<-rpois(no,spv[i])}
    
%%     %sam.cor<-cov2cor(cov(sam[1:100,1:100]))
%%     %for (i in 1:ncol(sam.cor)){
%%     %	for(k in 1:nrow(sam.cor)){
%%     %	if (abs(sam.cor[k,i])>=0.19){sam.cor[k,i]<-1}else{sam.cor[k,i]<-0}}}
%%     %gplot(sam.cor,gmode="graph")
%%     %@
%%     \caption{Example of a potential graph of species interactions
%%       generated through simulation by random number draws. Each
%%       species is represented by a node (dots) and the interactions
%%       between species are represented by edges (lines). This sort of
%%       information is not typically incorporated into community models
%%       or analyses.}
%%   \end{figure}
%% \end{center}

%% \subsection{Network Statistics}
%% One way that Network Theory could be useful in this context is that it
%% provides statistics (e.g. connectedness, degree and clustering) that
%% describe the whole system. For instance, by looking at the frequency
%% distribution of the number of interactions among species, we would
%% have some indication of the overall structure of the
%% species-interaction network (i.e. whether it follows a gaussian or
%% power-law distribution). 


%% \subsection{Network Models}
%% The current model of community evolution on a foundation species
%% (Shuster et al. 2006) does not incorporate interactions among the
%% species in the community. It may be possible to incorporate
%% interspecific interactions into this model in order to theoretically
%% evaluate the role that the structure of the species-ineraction network
%% has on how traits of a foundation species shape their associated
%% community and increase the robustness of these models for use in
%% testing for patterns in community data.


%% \subsection{Species-Interaction Network Data}
%% One important problem with working with ecological networks in
%% general, is obtaining the data. First, there is a problem with
%% defining the network itself. This is straightforward in some
%% situations. For instance, food-webs have a very simple
%% definintion. One thing eats another, and this is an interaction
%% (i.e. an edge). Other species-interaction networks are not so easily
%% defined. Of course the network can be defined in anyway that fits with
%% our knowledge of biology, but this definition will need to be
%% supported with valid arguments and the assumptions of the definition
%% will need to be acknowledged. 

%% \paragraph{}
%% Second, the networks typically used in network analyses are easily
%% observed (e.g. road networks, the internet and the
%% world-wide-web). That is to say, either the "graph" information
%% (i.e. the nodes and edges) is readily available or it can be easily
%% obtained via some automated process (e.g. web crawlers in the case of
%% the internet). Ecological networks are to not lend themselves so
%% easily to observation.  There are several possible ways to address
%% this problem:

%% \begin{enumerate}
%% \item{Exhaustively sample edges}
%% \item{Subsample the network (randomly or systematically)}
%% \item{Infer network structure from species abundance data}
%% \end{enumerate}

%% \section{Dissertation Chapters}
%% \begin{itemize}
%% \item{Chapter 1 - Review: Network Theory and Community Ecology}
%% \item{Chapter 2 - (Theory) Detecting Foundation Species Using Network
%%   Theory}
%% \item{Chapter 3 - (Modeling) Modeling Communities as Ecological
%%   Networks}
%% \item{Chapter 4 - (Data) Empirical Validation of Community Network
%%   Models}

%% \end{itemize}

\end{document}

