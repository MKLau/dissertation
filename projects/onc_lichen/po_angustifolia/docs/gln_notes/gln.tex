\documentclass[12pt]{article}
\usepackage{color}
\usepackage{cite}
\usepackage{geometry}                % See geometry.pdf to learn the layout options. There are lots.
%\usepackage{pdflscape}        %single page landscape
                                %mode \begin{landscape} \end{landscape}
\geometry{letterpaper}                   % ... or a4paper or a5paper or ... 
%\usepackage[parfill]{parskip}    % Activate to begin paragraphs with an empty line rather than an indent
\usepackage{graphicx}
\usepackage{amssymb}
\usepackage{/Library/Frameworks/R.framework/Resources/share/texmf/Sweave}


\title{Genetics of Lichen Networks}
\author{M.K. Lau}
%\date{}                                           % Activate to display a given date or no date

\begin{document}
\maketitle


\setcounter{tocdepth}{3}
\tableofcontents

\section{Summary}
\begin{itemize}
\item Intraspecific variation is an important contributor to
  ecological diversity
\item It is still not clear whether or not this is direct by
  influcening species abundances directly or if this is also through
  indirect effects
\item The complex structure and dynamics of communities is a potential
  source of unpredicted variation
\item Here, we present the results of a study of lichen species
  associated with cottonwood trees in a common garden and show that:
  \begin{enumerate}
  \item There is structure of co-occurrence patterns even in a small
    community of species (9 total)
  \item The structure of covariance networks varied signficantly among
    genotypes
  \end{enumerate}
\end{itemize}

\section{Data}

